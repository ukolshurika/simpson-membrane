\chapter{Математические модели}
Решение рассматриваемой задачи, основанное на степенной связи интенсивности напряжений и скоростей деформации ползучести при напряжениях, не превосходящих предела текучести материала, представлено в монографии Л.М.Качанова [\ref{kachanov}].
Решение задачи о деформировании мембраны в стесненных условиях при учете упрочнения материала приведены в монографиях Н.Н.~Малинина, [\ref{malinin}] и К.И. Романова [\ref{romanov}]. Известные работы [\ref{malinin},\ref{jerebcov}] допускают появление нефизичных бесконечных напряжений ($\sigma_u \to \infty$) в начальный момент времени, для их исключения в данной работе дополнительно учитывается мгновенное деформирование.
В [\ref{teraud_dis}, \ref{teraud}] приведено решение рассматриваемой задачи при различных граничных условиях, однако только для клиновидной матрицы. В данной работе приводится обобщение результатов, полученных в [\ref{teraud_dis}, \ref{teraud}], на случай криволинейной матрицы. Напряженное состояние мембраны можно считать безмоментным. Поскольку длина мембраны значительно превосходит её ширину, можно считать, что реализуется случай плоской деформации.

\section{Деформирование внутри криволинейной матрицы}
	\subsection{Первая стадия}
	Упругое деформирование мембраны описывается помощью закона Гука при сложном напряженном состоянии при учете несжимаемости материала мембраны.
	
	Вводим безразмерные переменные:
	\begin{equation}
		\overline{q} = \dfrac{q}{\sigma_b}, \;
		\overline{H} = \dfrac{H}{H_0}, \;
		\overline{H}_0 = \dfrac{H_0}{a}, \;
		\overline{t} = \dfrac{\sqrt 3}{2}Ct,\;
		k = \dfrac{E}{\sigma_b},\;
		\overline{\rho} = \dfrac{\rho}{H_0},
	\end{equation}
	где где $H$ ~--- толщина мембраны в произвольный момент времени, $H_0$~--- толщина мембраны при $t = -0$, $q$~--- давление, $2a$~--- ширина,
	$E$~--- модуль Юнга, $\rho$ толщина и радиус кривизны мембраны.
	При дальнейшем анализе черточки над безразмерными величинами опустим. Так как вывод формул описан в работах  [\ref{teraud_dis}, \ref{teraud}]
	и совпадает с нашим решением, приведем только итоговый результат, описывающий связь давления $q$ и мгновенно появляющегося угла $\alpha_1$,
	также приведены характеристики, описывающие состояние мембраны($H_1$~--- толщины и $\sigma_{\theta1}$):

	\begin{equation}
	\begin{split}
		q = \dfrac43 H_0k\sin(\alpha_1)\left(1-\dfrac{\sin(\alpha_1)}{\alpha_1} \right), \\
		H_1 = \dfrac{\sin(\alpha_1)}{\alpha_1},\\
		\sigma_{\theta1} = \dfrac{q}{H_1H_0\sin(\alpha_1)},
	\end{split}
	\label{guk_deformation}
	\end{equation}
	
	Следует отметить, что соотношения (\ref{guk_deformation}) позволяют исключить бесконечные напряжения в начальный момент времени.
	Так же следует отметить, что в работе расчет этих значений для материала были произведены с помощью программного средства Maxima.
	 
	\subsection{Вторая стадия}
	
	\subsection{Третья стадия}
\section{Деформирование внутри матрицы с вертикальными стенками и плоским днищем}
	\subsection{Первая стадия}
	\subsection{Вторая стадия}
	\subsection{Третья стадия}