Работа посвящена разработке и численной реализации модели деформирования длинной узкой 
прямоугольной мембраны внутри криволинейной матрицы и матрицы с вертикальными стенками и плоским днищем
при двух типах граничных условий: идеальное скольжение и прилипание.
При моделировании процесса ползучести использовалась дробно-сингулярная
модель ползучести, для численных расчетов применялись методы Симпсона и Гаусса.

В работе впервые решается задача о деформировании длинной узкой прямоугольной мембраны внутри криволинейной матрицы и матрицы с вертикальными стенками и плоским днищем при двух граничных условиях: идеальное скольжение и прилипание. Особенностью представленного решения является использование дробно-сингулярной модели установившейся ползучести, позволяющей учитывать возможное разрушение мембраны. Для задачи с П-образной матрицей 
(с вертикальными стенками и плоским днищем) характерной чертой является возникновение <<полостей>> при деформировании: при касании днища матрицы
происходит возникновение второй точки соприкосновения матрицы и мембраны, то есть образуется полость. До настоящего времени задача в данной постановке не рассматривалась.

Работа содержит \pageref{LastPage} страниц, 22 иллюстрации, 3 приложения.



Ключевые слова: ползучесть, деформирование, мембрана, разработка программного обеспечивания, численное интегрирование.
\newpage