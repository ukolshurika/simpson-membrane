Работа посвящена разработке и численной реализации модели деформирования длинной узкой 
прямоугольной мембраны внутри криволинейной матрицы и матрицы с вертикальными стенками и плоским днищем
при двух типах граничных условий: идеальное скольжение и прилипание.
При моделировании процесса ползучести использовалась дробно-сингулярная
модель ползучести, для численных расчетов применялись методы Симпсона и Гаусса.

Работа содержит \pageref{LastPage} страниц, 20 иллюстраций, 1 приложение.



Ключевые слова: ползучесть, деформирование, мембрана, разработка программного обеспечивания, численное интегрирование.
\newpage