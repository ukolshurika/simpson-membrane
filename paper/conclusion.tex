\chapter*{Заключение}
\addcontentsline{toc}{chapter}{Заключение}
В работе были проанализированы основные подходы к решению задач деформирования мембраны в стесненных условиях, учтены особенности подходов к решению и применяемые модели.  Моделирование деформирования длинной узкой прямоугольной мембраны 
внутри криволинейной матрицы и матрицы с вертикальными стенками и плоским днищем при двух различных граничных условиях было проведено при учете 
дробно-сингулярной модели установившейся ползучести, что отличает это решение от широко известных, где применяется простая, степенная модель установившейся ползучести.

Были получены основные соотношения, описывающие зависимость параметров состояния мембраны от времени, так
же были получены численные показатели данного процесса. Формально можно представить работу как решение двух независимых задач: задача о деформировании мембраны внутри криволинейной матрицы и задачу о деформировании мембраны внутри матрицы с вертикальными стенками и плоским днищем.
Для обеих задач учитывалось два типа граничных условий: идеальное скольжение и прилипание.

Разработан комплекс программ, в которых приведено применение  различных методов численного интегрирования для построения зависимости параметров деформации от времени, выбранных для обеспечения наилучшей точности. В качестве графических результатов расчета представлены не только графики зависимостей основных характеристик, но и видео с деформированием мембраны. В рамках работы была разработана программа моделирования ползучести мембраны при описанных условиях, так же было выполнено анимированное деформирование мембраны для случая идеального скольжения внутри матрицы с вертикальными стенками и плоским днищем.
