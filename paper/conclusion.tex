\chapter*{Заключение}
\addcontentsline{toc}{chapter}{Заключение}
В данной работе был проведен анализ существующих решений задачи о деформировании 
мембраны, в ходе которого была выбрана для реализации не обычная, широко применяемая 
степенная модель ползучести, а более сложная, дробно-сингулярная модель, позволяющая 
учитывать разрушение мембраны.

Впервые рассмотрены решения задач о деформировании длинной узкой прямоугольной мембраны в заданных условиях:
\begin{itemize}
\item[1.] внутри криволинейной матрицы,
\item[2.] внутри матрицы с вертикальными стенками и плоским днищем.
\end{itemize}
Отличительной особенностью решения второй задачи от существующих решений, является
не только модель, в которой возможен учет разрушения, но и то, что перемещение крайней точки касания мембраны является
 не монотонным (точка касания в начале двигается по вертикальной стенке, а потом касается днища, образуя две полости). В качестве граничных условий рассмотрены и промоделированы два типа, для каждой из задач:
\begin{itemize}
\item[1.] идеальное скольжение,
\item[2.] прилипание.
\end{itemize}

\medskip
Разработан комплекс программ, состоящий из модулей:
\begin{itemize}
\item[1.] вычислительное ядро,
\item[2.] подпрограмма построения графиков, 
\item[3.] подпрограмма визуализации,
\end{itemize}
 в которых приведено применение  различных методов 
численного интегрирования для построения зависимости параметров деформации от времени, 
выбранных для обеспечения наилучшей точности. В качестве графических результатов 
расчета представлены не только графики зависимостей основных характеристик, но и видео 
с деформированием мембраны.
