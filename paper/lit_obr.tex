\section{Постановка задачи}
Исследуется деформирование длинной узкой прямоугольной мембраны внутри криволинейной матрицы под действием равномерного поперечного 
давления $q$. При этом предполагаются закрепление мембраны вдоль ее длинных сторон и идеальное скольжение поверхности мембраны относительно поверхности матрицы.
Рассматривается 2 вида матриц: криволинейная парабола вида $y = b(1-x)^k$, с показателем степени $1<k \leqslant 2$, и матрица с вертикальными стенками и плоским днищем.

\section{Литературный обзор}

До настоящего момента существует множество решений задач о деформировании мембраны.
Решение задачи о свободном деформировании мембраны шириной $2l$ и начально толщиной $h_0$, закрепленной с двух длинных сторон и нагруженной равномерным давлением $p$ приведено в монографии Н.Н. Малинина [\ref{malinin}] 
При его решении была взята модель упрочнения:

$$\sigma_e = a\xi_e^{m_1}\P^{m_2}$$

В результате была получена зависимость угла раствора мембраны от времени:
$$\int\limits^0_tp^{1/m_1}\,dt = b\int\limits\Phi\,da;$$

$$\Phi  = (\dfrac{\sin^2\alpha}{\alpha})^{1/m_1}\left[\ln(\dfrac{\alpha}{\sin\alpha})\right]^{m_2/m_1}
\left(\dfrac{1}{a} - \ctg \alpha \right);$$

$$ b = \dfrac{2}{\sqrt 3}^{(m_1+m_2+1)/m_1}(ah_0/l)^{1/m_1}$$


\newpage