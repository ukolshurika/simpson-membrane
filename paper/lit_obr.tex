\section{Постановка задачи}
Исследуется деформирование длинной узкой прямоугольной мембраны внутри криволинейной матрицы под действием равномерного поперечного 
давления $q$. При этом предполагаются закрепление мембраны вдоль ее длинных сторон и идеальное скольжение поверхности мембраны относительно поверхности матрицы.
Рассматривается 2 вида матриц: криволинейная парабола вида $y = b(1-x)^k$, с показателем степени $1<k \leqslant 2$, и матрица с вертикальными стенками и плоским днищем.

\section{Литературный обзор}

При рассмотрении задачи о стесненном деформировании мембраны рассматривается 
две стадии деформирования: свободное деформирование (происходит вплоть до касания мембраной стенок матрицы) и стесненное деформирование( от касания мембраной стенок матрицы до заполнения или разрушешия мембраны). До настоящего момента существует множество решений задач о деформировании мембраны.

\subsection{Задача свободного деформирования, при учете упрочнения материала}
Решение задачи о свободном деформировании мембраны шириной $2l$ и начально толщиной $h_0$, закрепленной с двух длинных сторон и нагруженной равномерным давлением $p$ приведено в монографии Н.Н. Малинина [\ref{malinin}], используя модель упрочнения материала:

\begin{equation}
	\sigma_e = a\xi_e^{m_1}\text{P}^{m_2},
\end{equation}

где P~-- параметр Удквиста, $\xi$ деформации ползучести, $a, \; m_1\; m_2$~-- параметры материала, 
являющиеся справочной информацией[\ref{malinin_37}, \ref{malinin_132}, \ref{malinin_136}], $\sigma_e$~--
интенсивность напряжения. 

В результате была получена зависимость угла раствора мембраны от времени:
%\begin{equation}

	$$\int\limits^0_tp^{1/m_1}\,dt = b\int\limits_0^{\alpha}\Phi\,d\alpha;$$
	$$\Phi  = \left(\dfrac{\sin^2\alpha}{\alpha}\right)^{1/m_1}\left[\ln(\dfrac{\alpha}{\sin\alpha})\right]^{m_2/m_1}
	\left(\dfrac{1}{a} - \ctg \alpha \right);$$
	$$b = \left(\dfrac{2}{\sqrt 3}\right)^{(m_1+m_2+1)/m_1}(ah_0/l)^{1/m_1}$$
%\end{equation}

\subsection{Задача стесненного деформирования, при условии упрочнения материала}
Рассмотрим решение задачи, в случае идеального скольжения мембраны вдоль стенок матрицы, приведенное в 
монографии Н.Н. Малинина [\ref{malinin}]:

%\begin{equation}

	$$\int\limits^{t_1}_tp^{1/m_1}\,dt = b\int\limits_0^x\Phi_1\,d\alpha;$$
	$$\Phi_1  = \dfrac{\chi_1}{\chi_2+\chi_1x}[\ln(\chi_2+\chi_1x)]^{\frac{m_2}{m_1}}
	\left\lbrace\dfrac{\alpha}{[\chi_2-(1-\chi_1)x](\chi_2+\chi_1x)}\right\rbrace^\frac{1}{m_1};$$
	$$b = \left(\dfrac{2}{\sqrt 3}\right)^{(m_1+m_2+1)/m_1}(ah_0/l)^{1/m_1};$$
	$$\chi_1=1-\alpha\ctg\alpha;\;\chi_2=\alpha\sin\alpha.$$
%\end{equation}

Из этого уравнения численно определяется зависимость безразмерной длины контакта от времени $x(t)$ для 
заданного закона изменения давления во времени. Зная длину участка контакта можно определить толщину 
мембраны и напряжения. Решения для случая прилипания также приведено в [\ref{malinin}].

 
\newpage