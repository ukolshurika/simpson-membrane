\documentclass{article}
\usepackage{tikz}
\usepackage[unicode]{hyperref}
\usepackage[utf8]{inputenc}
\usepackage[russian]{babel}
\usepackage{pgfplots}
\usepackage{epsfig,anysize,amssymb,verbatim,hhline,texnames,subfigure,multicol, floatflt, amsmath}
\usepackage[hmargin=3cm,vmargin=2cm]{geometry}
\usetikzlibrary{external}
\usetikzlibrary{patterns}
\tikzset{external/system call={latex \tikzexternalcheckshellescape -halt-on-error
-interaction=batchmode -jobname "\image" "\texsource";
dvips -o "\image".ps "\image".dvi;
ps2eps "\image.ps"}}
\tikzexternalize
\begin{document}

% \begin{tikzpicture}
%   \begin{axis}[
%     scale only axis,
%       xlabel=$\overline{t} \cdot 10^{-8}$,
%       ylabel=$\overline{H}(t)$,
%       axis x line=box,
%     axis y line*=left,
%       ytick={0,0.2,...,0.8, 0.97},
%       xtick={0,5.36,15, 30,...,80},
%       ymin=0,
%       ymax=1,
%       xmin=-0.1,
%       xmax=80]
%       \addplot[mark=none, smooth] file {data/h_is_1.5.out};   % Here is the data file
%       \addplot[mark=none, dotted] coordinates {(5.36, 0) (5.36, 1)};   % Here is the data file
%   \end{axis}
%   \begin{axis}[   
%     scale only axis,
%       ylabel=$\overline{\sigma}_u(t) \cdot 10^3$,
%       axis y line*=right,
%       y label style={at={(1.26,0.5)}},
%     axis x line=center,
%       ytick={0,10,...,20},
%       xtick={0,15,...,80},
%       ymin=0,
%       ymax=20,
%       xmin=-0.1,
%       xmax=80]
%     \addplot[mark=none,smooth, dashdotted] file {data/s_is_1.5.out};   % Here is the data file 
%     \addplot[mark=none, dotted] coordinates {(5.36, 0) (5.36, 20)}; 
%     %\addplot[mark=none] coordinates {(60, 16) (60, 19.5) (75, 19.5) (75, 16) (60, 16)};
%     \addplot[mark=none, dashdotted] coordinates {(57, 17) (70, 17)};
%     \addplot[mark=none] coordinates {(57, 18.5) (70, 18.5)};
%     \node at (72, 170) {2};
%     \node at (72, 185) {1};
%   \end{axis}
% \end{tikzpicture}


% \begin{tikzpicture}
%   \begin{axis}[
%     scale only axis,
%       xlabel=$x$,
%       ylabel=$y$,
%       samples={500},
%       axis x line=center,
%     axis y line=center,
%       ytick={0},
%       xtick={0},
%       xmin=-1.2,
%       xmax=1.2, 
%       ymax=7.5,
%       ymin=-0.5 ]
%       \addplot[mark=none, color=black] gnuplot[id=mtrx,domain=-1:1]{6.5*(1-abs(x)**(1.5))};
%     \node at (220, 26) {a};
%     \node at (116, 720) {b};
%     \draw[pattern=north west lines, pattern color=gray, line width=0.2pt] (20,37) rectangle (220,49);
%   \end{axis}
  
% \end{tikzpicture}
\begin{figure}
  \centering
  \begin{tikzpicture}
    \begin{axis}[
      scale only axis,
        xlabel=$\overline{t} \cdot 10^{-8}$,
        ylabel=$\overline{H}(t)$,
        axis x line=box,
      axis y line*=left,
        ytick={0,0.2,...,0.8, 0.97},
        % xtick={0,7.14,15, 25,...,80},
        ymin=0,
        ymax=1,
        xmin=-0.1,
        xmax=80]
        \addplot[mark=none, smooth] file {data/h_cup.dat};   % Here is the data file
    \end{axis}
    \begin{axis}[   
      scale only axis,
        ylabel=$\overline{\sigma}_u(t) \cdot 10^3$,
        axis y line*=right,
        y label style={at={(1.26,0.5)}},
        ytick={0,10,...,20},
        xtick={},
        ymin=0,
        ymax=20,
        xmin=-0.1,
        xmax=80]
      \addplot[mark=none,smooth, dashdotted] file {data/s_cup.dat};   % Here is the data file 
      \addplot[mark=none, dotted] coordinates {(7.14, 0) (7.14, 20)}; 
      \addplot[mark=none, dashdotted] coordinates {(57, 17) (70, 17)};
      \addplot[mark=none] coordinates {(57, 18.5) (70, 18.5)};
      \node at (72, 170) {2};
      \node at (72, 185) {1};
    \end{axis}
  \end{tikzpicture}
  \caption{$b$=4.5$a$}
\end{figure}

\begin{figure}
  \centering
  \begin{tikzpicture}
    \begin{axis}[
      scale only axis,
        xlabel=$\overline{t} \cdot 10^{-8}$,
        ylabel=$\overline{H}(t)$,
        axis x line=box,
      axis y line*=left,
        ytick={0,0.2,...,0.8, 0.97},
        % xtick={0,7.14,15, 25,...,80},
        ymin=0,
        ymax=1,
        xmin=-0.1,
        xmax=80]
        \addplot[mark=none, smooth] file {data/h2_cup.dat};   % Here is the data file
    \end{axis}
    \begin{axis}[   
      scale only axis,
        ylabel=$\overline{\sigma}_u(t) \cdot 10^3$,
        axis y line*=right,
        y label style={at={(1.26,0.5)}},
        ytick={0,10,...,20},
        xtick={},
        ymin=0,
        ymax=20,
        xmin=-0.1,
        xmax=80]
      \addplot[mark=none,smooth, dashdotted] file {data/s2_cup.dat};   % Here is the data file 
      \addplot[mark=none, dotted] coordinates {(7.14, 0) (7.14, 20)}; 
      \addplot[mark=none, dashdotted] coordinates {(57, 17) (70, 17)};
      \addplot[mark=none] coordinates {(57, 18.5) (70, 18.5)};
      \node at (72, 170) {2};
      \node at (72, 185) {1};
    \end{axis}
  \end{tikzpicture}
  \caption{$b$=$a$}
\end{figure}

\begin{figure}
  \centering
  \begin{tikzpicture}
    \begin{axis}[
      scale only axis,
        xlabel=$\overline{t} \cdot 10^{-8}$,
        ylabel=$\overline{H}(t)$,
        axis x line=box,
      axis y line*=left,
        ytick={0,0.2,...,0.8, 0.97},
        % xtick={0,7.14,15, 25,...,80},
        ymin=0,
        ymax=1,
        xmin=-0.1,
        xmax=80]
        \addplot[mark=none, smooth] file {data/h3_cup.dat};   % Here is the data file
    \end{axis}
    \begin{axis}[   
      scale only axis,
        ylabel=$\overline{\sigma}_u(t) \cdot 10^3$,
        axis y line*=right,
        y label style={at={(1.26,0.5)}},
        ytick={0,10,...,20},
        xtick={},
        ymin=0,
        ymax=20,
        xmin=-0.1,
        xmax=80]
      \addplot[mark=none,smooth, dashdotted] file {data/s3_cup.dat};   % Here is the data file 
      \addplot[mark=none, dotted] coordinates {(7.14, 0) (7.14, 20)}; 
      \addplot[mark=none, dashdotted] coordinates {(57, 17) (70, 17)};
      \addplot[mark=none] coordinates {(57, 18.5) (70, 18.5)};
      \node at (72, 170) {2};
      \node at (72, 185) {1};
    \end{axis}
  \end{tikzpicture}
  \caption{$b$=10$a$}
  Уже тут получается белиберда: после определенного момента времени 
  толщина становися отрицательной (думаю так на численных данных 
  сказывается разрушение мембраны, до момента касания днища). Хотя разрушение мембраны вполне разумно: 
  В случае клиновидной матрицы у нас уменьшался радиус дуги свободно деформируемой части, а здесь нет уменьшающихся характеристик деформированности мембраны, только её утоньшение, поэтому, наверное, разрушение все таки возможно. 
\end{figure}

\begin{figure}
  \centering
  \begin{tikzpicture}
    \begin{axis}[
      scale only axis,
        xlabel=$\overline{t} \cdot 10^{-8}$,
        ylabel=$\overline{H}(t)$,
        axis x line=box,
      axis y line*=left,
        ytick={0,0.2,...,0.8, 0.97},
        % xtick={0,7.14,15, 25,...,80},
        ymin=0,
        ymax=1,
        xmin=-0.1,
        xmax=80]
        \addplot[mark=none, smooth] file {data/h4_cup.dat};   % Here is the data file
    \end{axis}
    \begin{axis}[   
      scale only axis,
        ylabel=$\overline{\sigma}_u(t) \cdot 10^3$,
        axis y line*=right,
        y label style={at={(1.26,0.5)}},
        ytick={0,10,...,20},
        xtick={},
        ymin=0,
        ymax=20,
        xmin=-0.1,
        xmax=80]
      \addplot[mark=none,smooth, dashdotted] file {data/s4_cup.dat};   % Here is the data file 
      \addplot[mark=none, dotted] coordinates {(7.14, 0) (7.14, 20)}; 
      \addplot[mark=none, dashdotted] coordinates {(57, 17) (70, 17)};
      \addplot[mark=none] coordinates {(57, 18.5) (70, 18.5)};
      \node at (72, 170) {2};
      \node at (72, 185) {1};
    \end{axis}
  \end{tikzpicture}
  \caption{$b$=7.05$a$ }
  Для 7.15 уже наступает разрушение, так что это почти предельое значение при котором мембрана успевает коснуться днища матрицы.
\end{figure}

\end{document}
